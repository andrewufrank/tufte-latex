\documentclass{tufte-book}

\hypersetup{colorlinks}% uncomment this line if you prefer colored hyperlinks (e.g., for onscreen viewing)

%%
% Book metadata
\title{Ein Buch mit Titel}
\author[The Tufte-LaTeX Developers]{AFrank}
\publisher{Gerastree}

%%
% If they're installed, use Bergamo and Chantilly from www.fontsite.com.
% They're clones of Bembo and Gill Sans, respectively.
%\IfFileExists{bergamo.sty}{\usepackage[osf]{bergamo}}{}% Bembo
%\IfFileExists{chantill.sty}{\usepackage{chantill}}{}% Gill Sans

%\usepackage{microtype}

%%
% Just some sample text
\usepackage{lipsum}
\usepackage[]{mdframed} %af added
%%
% For nicely typeset tabular material
\usepackage{booktabs}

%%
% For graphics / images
\usepackage{graphicx}
\setkeys{Gin}{width=\linewidth,totalheight=\textheight,keepaspectratio}
\graphicspath{{graphics/}}

% The fancyvrb package lets us customize the formatting of verbatim
% environments.  We use a slightly smaller font.
\usepackage{fancyvrb}
\fvset{fontsize=\normalsize}

%%
% Prints argument within hanging parentheses (i.e., parentheses that take
% up no horizontal space).  Useful in tabular environments.
\newcommand{\hangp}[1]{\makebox[0pt][r]{(}#1\makebox[0pt][l]{)}}

%%
% Prints an asterisk that takes up no horizontal space.
% Useful in tabular environments.
\newcommand{\hangstar}{\makebox[0pt][l]{*}}

%%
% Prints a trailing space in a smart way.
\usepackage{xspace}

%%
% Some shortcuts for Tufte's book titles.  The lowercase commands will
% produce the initials of the book title in italics.  The all-caps commands
% will print out the full title of the book in italics.
\newcommand{\vdqi}{\textit{VDQI}\xspace}
\newcommand{\VDQI}{\textit{The Visual Display of Quantitative Information}\xspace}

\newcommand{\TL}{Tufte-\LaTeX\xspace}

% Prints the month name (e.g., January) and the year (e.g., 2008)
\newcommand{\monthyear}{%
  \ifcase\month\or January\or February\or March\or April\or May\or June\or
  July\or August\or September\or October\or November\or
  December\fi\space\number\year
}


% Prints an epigraph and speaker in sans serif, all-caps type.
\newcommand{\openepigraph}[2]{%
  %\sffamily\fontsize{14}{16}\selectfont
  \begin{fullwidth}
  \sffamily\large
  \begin{doublespace}
  \noindent\allcaps{#1}\\% epigraph
  \noindent\allcaps{#2}% author
  \end{doublespace}
  \end{fullwidth}
}

% Inserts a blank page
\newcommand{\blankpage}{\newpage\hbox{}\thispagestyle{empty}\newpage}

\usepackage{units}

% Typesets the font size, leading, and measure in the form of 10/12x26 pc.
\newcommand{\measure}[3]{#1/#2$\times$\unit[#3]{pc}}

% Macros for typesetting the documentation
\newcommand{\hlred}[1]{\textcolor{Maroon}{#1}}% prints in red
\newcommand{\hangleft}[1]{\makebox[0pt][r]{#1}}
\newcommand{\hairsp}{\hspace{1pt}}% hair space
\newcommand{\hquad}{\hskip0.5em\relax}% half quad space
\newcommand{\TODO}{\textcolor{red}{\bf TODO!}\xspace}
\newcommand{\na}{\quad--}% used in tables for N/A cells
\providecommand{\XeLaTeX}{X\lower.5ex\hbox{\kern-0.15em\reflectbox{E}}\kern-0.1em\LaTeX}
\newcommand{\tXeLaTeX}{\XeLaTeX\index{XeLaTeX@\protect\XeLaTeX}}
% \index{\texttt{\textbackslash xyz}@\hangleft{\texttt{\textbackslash}}\texttt{xyz}}
\newcommand{\tuftebs}{\symbol{'134}}% a backslash in tt type in OT1/T1
\newcommand{\doccmdnoindex}[2][]{\texttt{\tuftebs#2}}% command name -- adds backslash automatically (and doesn't add cmd to the index)
\newcommand{\doccmddef}[2][]{%
  \hlred{\texttt{\tuftebs#2}}\label{cmd:#2}%
  \ifthenelse{\isempty{#1}}%
    {% add the command to the index
      \index{#2 command@\protect\hangleft{\texttt{\tuftebs}}\texttt{#2}}% command name
    }%
    {% add the command and package to the index
      \index{#2 command@\protect\hangleft{\texttt{\tuftebs}}\texttt{#2} (\texttt{#1} package)}% command name
      \index{#1 package@\texttt{#1} package}\index{packages!#1@\texttt{#1}}% package name
    }%
}% command name -- adds backslash automatically
\newcommand{\doccmd}[2][]{%
  \texttt{\tuftebs#2}%
  \ifthenelse{\isempty{#1}}%
    {% add the command to the index
      \index{#2 command@\protect\hangleft{\texttt{\tuftebs}}\texttt{#2}}% command name
    }%
    {% add the command and package to the index
      \index{#2 command@\protect\hangleft{\texttt{\tuftebs}}\texttt{#2} (\texttt{#1} package)}% command name
      \index{#1 package@\texttt{#1} package}\index{packages!#1@\texttt{#1}}% package name
    }%
}% command name -- adds backslash automatically
\newcommand{\docopt}[1]{\ensuremath{\langle}\textrm{\textit{#1}}\ensuremath{\rangle}}% optional command argument
\newcommand{\docarg}[1]{\textrm{\textit{#1}}}% (required) command argument
\newenvironment{docspec}{\begin{quotation}\ttfamily\parskip0pt\parindent0pt\ignorespaces}{\end{quotation}}% command specification environment
\newcommand{\docenv}[1]{\texttt{#1}\index{#1 environment@\texttt{#1} environment}\index{environments!#1@\texttt{#1}}}% environment name
\newcommand{\docenvdef}[1]{\hlred{\texttt{#1}}\label{env:#1}\index{#1 environment@\texttt{#1} environment}\index{environments!#1@\texttt{#1}}}% environment name
\newcommand{\docpkg}[1]{\texttt{#1}\index{#1 package@\texttt{#1} package}\index{packages!#1@\texttt{#1}}}% package name
\newcommand{\doccls}[1]{\texttt{#1}}% document class name
\newcommand{\docclsopt}[1]{\texttt{#1}\index{#1 class option@\texttt{#1} class option}\index{class options!#1@\texttt{#1}}}% document class option name
\newcommand{\docclsoptdef}[1]{\hlred{\texttt{#1}}\label{clsopt:#1}\index{#1 class option@\texttt{#1} class option}\index{class options!#1@\texttt{#1}}}% document class option name defined
\newcommand{\docmsg}[2]{\bigskip\begin{fullwidth}\noindent\ttfamily#1\end{fullwidth}\medskip\par\noindent#2}
\newcommand{\docfilehook}[2]{\texttt{#1}\index{file hooks!#2}\index{#1@\texttt{#1}}}
\newcommand{\doccounter}[1]{\texttt{#1}\index{#1 counter@\texttt{#1} counter}}

% Generates the index
\usepackage{makeidx}
\makeindex

\begin{document}

% Front matter
\frontmatter

% r.1 blank page
\blankpage

% v.2 epigraphs
\newpage\thispagestyle{empty}

\vfill
\openepigraph{%
Was soll man tun?
}{AF}
\vfill



% r.3 full title page
\maketitle


% v.4 copyright page
\newpage
\begin{fullwidth}
~\vfill
\thispagestyle{empty}
\setlength{\parindent}{0pt}
\setlength{\parskip}{\baselineskip}
Copyright \copyright\ \the\year\ \thanklessauthor

\end{fullwidth}

% r.5 contents
\tableofcontents

\listoffigures

\listoftables






%%
% Start the main matter (normal chapters)
\mainmatter

Here the main matter
Design of \texttt{daino} was guided by some principles which are
explained in this \texttt{ReadMe}.


\chapter{No proprietary file formats}
\begin{mdframed}The source of the web site is formatted with common, non-proprietary formats.\end{mdframed}
The source of the web site should be stored in open file formats, which
can be exchanged and read by many programs. It should be easy to take a
site organized with one SSG and put it into another one. Proprietary
formats make it typically hard to extract content, store it in an open
format, and to move it to another program -\/-\/- effectively locking
users in.

The principle speaks against use of databases to store content (so
called Content Management Systems) which are probably justified for
large, very high traffic sites, a use case, \texttt{daino} is not design
for.

\texttt{Daino} organized the source for a web site in text files written
in Markdown; they can be edited with any simple text editor\footnote{Using
	a \emph{intelligent} editor like Word is inconvenient;
	\texttt{VS\ code} however works well.}.

\chapter{Daino organizes a site as a tree}
\begin{mdframed}The web pages are structured as a tree and collected in a directory tree.\end{mdframed}
\hypertarget{principle-the-structure-of-the-site-and-the-structure-of-it-is-stored-representation-should-correspond}{%
	\section{Principle: The structure of the site and the structure of it is
		stored representation should
		correspond}\label{principle-the-structure-of-the-site-and-the-structure-of-it-is-stored-representation-should-correspond}}

A web site is presented as pages of hyper-text with links between the
pages\citep{berners2001semantic}. This logical structure is represented
as files and the whole site is collected under a root directory.

The mapping between rendered web pages and the files representing them
is crucial in the design:

\textbf{Each web page is stored as a markdown file.}\footnote{Additional
	material can be stored in files in a \texttt{resources} directory.}

Each web page in a site is written as a markdown file, which the
generator transforms to a HTML file which can be rendered. The structure
of the source (\texttt{dough}) of the web page is parallel to the
directory structure of the \texttt{baked} homepage, which can be served
by a web server and rendered by a browser.

A markdown page can call for \textbf{additional material} and link to
other renderable pages not produced from a markdown page.

\hypertarget{tree-structure}{%
	\subsection{Tree structure}\label{tree-structure}}

The web site starts with a single page\footnote{Often called
	\texttt{landing\ page}.} from which all other pages can reached in a
tree structure.

The web pages are stored as files in directories. The directory tree
starts with the root (here \texttt{dainoSite/dough}) which contains all
the source text for the web pages\footnote{It contains an additional
	file \texttt{settingsNN.yaml}, currently \texttt{settings3.yaml} for
	the site.}.

Directories store only files and additional information for the
presentation of the directory as web page is necessary. For each
directory an \texttt{index.md} file is added which comments on the
directories content and the list of directories is rendered.

Additional content can be stored in \texttt{resources}
directories\footnote{Which must be called \texttt{resources}, allother
	directories are assumed to be conent directories!}

\hypertarget{correspondence-between-presentation-and-storage}{%
	\subsection{Correspondence between presentation and
		storage}\label{correspondence-between-presentation-and-storage}}

The source for web pages, and the web pages in \texttt{HTML} formate are
stored in a parallel directory structure and correspond to the structure
of the web site visible to the user.

\chapter{Pandoc converts from markdown to HTML}
\begin{mdframed}The sources of the web pages are (primarily) written as markdown and converted by Pandoc to HTML.\end{mdframed}
\hypertarget{source-files-are-converted-to-html-using-pandoc}{%
	\section{Source files are converted to HTML using
		Pandoc}\label{source-files-are-converted-to-html-using-pandoc}}

The web page sources are translated using Pandoc to HTML and PDF. Pandoc
is equally used to convert the markdown sources to latex and then to
PDF.

Pandoc would allow three dozens of
\href{https://pandoc.org/MANUAL.html}{input formats}. At the moment,
page sources must be written in the Pandoc markdown language, but
essentially any other input Pandoc can read could be used (e.g.
\texttt{latex}).

\hypertarget{shake-controls-the-conversion}{%
	\section{\texorpdfstring{\texttt{Shake} controls the
			conversion}{Shake controls the conversion}}\label{shake-controls-the-conversion}}

Shake is an improved \texttt{make} producing a desired set of files from
sources and rules. It is driven by the correspondence between the
\texttt{md} files which must be converted to \texttt{HTML} and draws in
additional files as necessary. It converts files only if changed; files
can be watched for changes and automatically converted.

\chapter{Help with language specific input}
\begin{mdframed}Various conventions to speed up textual input exist and can be supported; currently support for german text input is built in.\end{mdframed}
In the \texttt{yaml\ header} of a markdown file the language must be
specified if not U.S. english (marked as
\texttt{en\_US).\ I\ use\ often}de\_CH` for the Swiss variants of
german.

Various conventions to type text in non-english languages exist. For
example, it is customary to type combinations of American keyboard
characters to stand in for those which are not found on the standard
American keyboard. For example, when typing a German text, often the
umlaut ä, ö and ü are written as "ae", "oe" and "ue". Unfortunately, it
is not possible to just use a global replace, because some German words
contain some such combinations (e.g. Koeffizienten) which must not be
written as Köffizienten! Similar conventions exist in other languages to
type, e.g. email on standard keyboards. Italians, for example, replace
an accented character with a appended apostrophe
(italianita\textquotesingle).

Daino includes a support program for german language writing which is
automatically applied to German texts and replaces umlauts when
acceptable. It uses a small list to guide the process, which is not
perfect. Omissions can be edited manually and are not affected by later
replacements. Commissions must be - on a file by file base - collected
in type YAML header as \texttt{doNotReplace} list\footnote{Specifically
	useful to allow some english words, like "blue", in a German text!}.
Such a list remains with the file and need only updated when text is
added and the replacement process produces undesired changes.

\chapter{Produce books}
\begin{mdframed}From a bunch of blog-sized posts a booklet or a book should be produced in a printable, i.e. pdf format.\end{mdframed}
A collection of blog posts, a.k.a. \texttt{md} files a comprehensive
printable version should be produced.

It is based on the construction of the index pages, which include
automatically all \texttt{md} files in the directory in which the
\texttt{index.md} file is placed.

Marking the \texttt{yaml\ header} of the index file with

\begin{verbatim}
	book: booklet 
\end{verbatim}

or

\begin{verbatim}
	book: bookbig
\end{verbatim}

a comprehensive file for the directory, respectively for the
\texttt{bookbig} case, the diretory and its subdirectories, which must
be marked with \texttt{booklet}, is produced. It uses \texttt{part} and
\texttt{chapter} for two levels of titles above the \texttt{section}
title marked with

\begin{verbatim}
	title: someText 
\end{verbatim}

in the \texttt{yaml\ header}. The single markdown header marker
\texttt{\#} is interpreted as \texttt{subsection}.

For \texttt{bookbig} and \texttt{booklet} table of content is
automatically produced.

\chapter{Markdown as primary input format}
\begin{mdframed}The web pages are written as markdown text, which allows emphasis, titles, references, images, footnotes etc.\end{mdframed}
Markdown is an easy to learn and versatile. The list possible formatting
is quite comprehensive:

"Markdown may not be the right format for you if you find these elements
not enough for your writing:

\begin{itemize}
%	\tightlist
	\item
	paragraphs,
	\item
	(section) headers,
	\item
	block quotations,
	\item
	code blocks,
	\item
	(numbered and unnumbered) lists,
	\item
	horizontal rules,
	\item
	tables,
	\item
	inline formatting (emphasis, strikeout, superscripts, subscripts,
	"booklet"im, and small caps text),
	\item
	LaTeX math expressions,
	\item
	equations,
	\item
	links,
	\item
	images,
	\item
	footnotes,
	\item
	citations,
	\item
	theorems,
	\item
	proofs, and
	\item
	examples."\footnote{\href{https://bookdown.org/yihui/rmarkdown/\#preface}{Bookdown}.}
\end{itemize}

In exceptional circumstances additional formatting tricks can be pulled
in as HTML code.

\hypertarget{yaml-header}{%
	\section{YAML header}\label{yaml-header}}

Markdown allows headers to pass metadata about a file (e.g. title,
author) in a \href{https://yaml.org/spec/1.2.2/}{YAML} to processes
working on the source text; the format is flexible\footnote{But beware
	of colons, quotes etc.!}.

\hypertarget{markdown-can-include-images-reference-etc}{%
	\section{Markdown can include images, reference
		etc.}\label{markdown-can-include-images-reference-etc}}

Markdown allow the inclusion of images, bibliographic references etc.
These additional files are stored in \texttt{resources}
directories\footnote{\texttt{resources} is a reserved word; all other
	directory names are treated as content directories}.

References are always \texttt{absolute} with respect to the
root\footnote{starting with a "/"} or \texttt{relative} to the current
page\footnote{not starting with "/"}.

\chapter{Separate content and presentation (aka theme)}
\begin{mdframed}The content of the web pages should be indpendent of the presentation.\end{mdframed}
\hypertarget{presentation-can-be-changed}{%
	\section{Presentation can be
		changed}\label{presentation-can-be-changed}}

A change in the presentation style should not affect the content. It
must bepossible to move from fixed-width presentation to a presentation
which adapts to different screen sized and later to use a Tufte-inspired
style without touching the web page content.

Markdown allows to structure the content with hints for the presentation
(e.g. title, footnote) but not fixing how these are rendered.

\hypertarget{theme-directory}{%
	\section{Theme directory}\label{theme-directory}}

The instructions for presentation, the so called \texttt{theme} is in a
separate directory (here `dainoSite/theme). It is linked automatically
into the baked site.

The theme consists of

\begin{itemize}
%	\tightlist
	\item
	fonts, preferably in the \texttt{woff} format,
	\item
	images
	\item
	cascading style sheets (CSS) in \texttt{static} folder.
\end{itemize}

The elements are brought together with the content using the
\href{https://hackage.haskell.org/package/pandoc-3.1.1/docs/Text-Pandoc-Templates.html}{Pandoc
	templates} and a template to construct a \texttt{LaTeX} input file to
produce the PDF.\footnote{Currently \texttt{master7tufte.dtpl} for HTML
	output and \texttt{latex7.dtpl} for PDF output.}

\chapter{Produced web site is self-contained}
\begin{mdframed}The Static Site generated is selfcontained. It can be served by any web server.\end{mdframed}
The files in the \texttt{baked} directory includes everything a web
server needs to access and is relocatable. It can be copied to become
the web root of a server.

Any web server to which a user can upload files to the web root can be
used.\footnote{I currently use a service giving my a \texttt{cpanel} to
	which I can upload with \texttt{ftp}; perhaps not the most convenient
	solution but sufficient.}





endcontentTest

\backmatter

\bibliography{sample-handout}
\bibliographystyle{plainnat}


\printindex

\end{document}

